\documentclass[conference]{IEEEtran}

\usepackage[brazil]{babel}
\usepackage[T1]{fontenc}
\usepackage[utf8]{inputenc}
\hyphenation{op-tical net-works semi-conduc-tor}


\begin{document}

\title{Resultados das simulações com The One}

\author{\IEEEauthorblockN{Thiago de Gouveia Nunes}
\IEEEauthorblockA{Instituto de Matemática e Estatística\\
Universidade São Paulo\\
São Paulo, São Paulo\\
Email: thiago.gouveia.nunes@gmail.com}
\and
\IEEEauthorblockN{Homer Simpson}
\IEEEauthorblockA{Twentieth Century Fox\\
Springfield, USA\\
Email: homer@thesimpsons.com}
\and
\IEEEauthorblockN{James Kirk\\ and Montgomery Scott}
\IEEEauthorblockA{Starfleet Academy\\
San Francisco, California 96678-2391\\
Telephone: (800) 555--1212\\
Fax: (888) 555--1212}}

\IEEEpeerreviewmaketitle



\section{Introduction}
  O objetivo desse trabalho é comparar protocolos de roteamento para DTNs usando a ferramenta de simulação
\textit{The One}. 4 cenários foram criados, e 5 simulações foram rodadas sobre cada cenário, cada uma com
um protocolo diferente. Com isso, podemos determinar quais protocolos tem vantagem sobre outros em situações
semelhantes as dos cenários.

\section{Conclusion}
The conclusion goes here.

\section*{Acknowledgment}


The authors would like to thank...

\begin{thebibliography}{1}

\bibitem{IEEEhowto:kopka}
H.~Kopka and P.~W. Daly, \emph{A Guide to \LaTeX}, 3rd~ed.\hskip 1em plus
  0.5em minus 0.4em\relax Harlow, England: Addison-Wesley, 1999.

\end{thebibliography}

\end{document}


